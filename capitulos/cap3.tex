% !TEX root = ../main.tex
\documentclass[../main.tex]{subfiles}

\chapter{Aprendizaje con Kernels}

  %$\bm{O}_{*} = \mathbf { 1 } _{N_{*}} \mathbf { 1 }_{N}^T /N$ \;
  %
  %$\tilde{ \mathbf { K } } = \bm{K}_{*} - \bm{O}_{*}\bm{K}_{*} - \bm{K}_{*}\bm{O}_{*} + \bm{O}_{*}\bm{K}_{*}\bm{O}_{*}$\;
  %
  %$Z = \tilde{ \mathbf { K } } \bm{V}(:,1:L)$;
\section{Introducción}

Un kernel es una función simétrica y definida positiva $k(\cdot,\cdot)$ que
puede ser entendida como una medida de similitud entre los argumentos que opera.
En el siguiente capítulo, se definen estos objetos matemáticos de manera formal,
se investigan sus características y se derivan algunos métodos del aprendizaje
de máquinas que toman ventaja sus propiedades.

\section{Terminología y propiedades}

El término \textbf{kernel} proviene del estudio de operadores integrales en el
campo del análisis funcional. En tal contexto se les identifica como aquellas
funciones $k$ que determinan un operador $T_k$ a través de:
\begin{equation}
    (T_k f) (x) = \int_{\mathcal{X}} k(x,x') f(x') dx
\end{equation}

En concordancia con la perspectiva que se desea abarcar, se denotará como kernel
a toda función $k: \mathcal{X} \times \mathcal{X} \rightarrow \mathbb{R}$ en el
espacio de características \footnote{El condominio del kernel $k$ no tiene por que
restringirse a los reales, para ciertas aplicaciones puede ser conviene utilizar
los complejos. En general las propiedades de interés se preservan en ambos cuerpos,
por lo que por simplicidad se considera solo el caso real en esta monografía.}.
Dentro de tal clase de funciones, son de importancia a aquellas capaces de ``generalizar"
el concepto de \textit{porducto interno}, el \index{Teorema de Mercer}
\textbf{Teorema de Mercer} permite identificar tal subconjunto.
\NC{Agregar apéndice sobre Teoria de Medida + integrar respecto a una medida
learning with Kernels}
\begin{theorem}[Teorema de Mercer]
\label{mercer}
Sea $(\mathcal{X},\mu)$ un espacio de medida finita y $k \in L_{\infty}(\mathcal{X}^2)$
una función real y simétrica, tal que el operador integral:

\begin{gather}
    \begin{aligned}
        T_k &: L_{2}(\mathcal{X}) \rightarrow L_{2}(\mathcal{X}) \\
        (T_k f) (x) &:= \int_{\mathcal{X}} k(x,x') f(x') d\mu(x')
    \end{aligned}
\end{gather}

es semidefinido positivo, es decir, para toda $f \in L_{2}(\mathcal{X})$ se cumple:
\begin{equation}
    \int_{\mathcal{X}^2} k(x,x') f(x) f(x') d\mu(x) d\mu(x') \geq 0
\end{equation}

Si $\psi_j \in L_2(\mathcal{X})$ son las funciones propias ortonormales de $T_f$
asociadas a los valores propios $\lambda_j > 0$, ordenados de manera decreciente.
Entonces,
\begin{enumerate}
    \item $(\lambda_j)_j \in l_1$
    \item $k(x,x')=\sum_{j=1}^{N_{\mathcal{H}}} \lambda_j \psi_j(x)\psi_j(x')$
    se cumple para casi todos los elementos $x,x' \in \mathcal{X}$. Además $N_{\mathcal{H}} \in \mathbb{N}$
    o $N_{\mathcal{H}} = \infty$, en este último caso, la serie respectiva converge
    absoluta y uniformemente para casi todos los elementos $x,x' \in \mathcal{X}$.
\end{enumerate}

\end{theorem}

La segunda implicación del teorema anterior, permite construir una aplicación
$\Phi: \mathcal{X} ~ \rightarrow ~ l_2^{N_{\mathcal{H}}}$ tal que
$x ~ \mapsto ~ (\sqrt{\lambda_j} \psi_j(x))_{j=1,\ldots,N_{\mathcal{H}}}$, es decir,
a cada elemento de $\mathcal{X}$ se le asocia una transformación por medio de las
funciones propias asociadas a $k(\cdot,\cdot)$ al espacio $l_2^{N_{\mathcal{H}}}$.
Este último espacio esta provisto de producto interno, por lo que es posible expresar
$k(x,x') = \langle  \Phi(x), \Phi(x') \rangle_{l_2^{N_{\mathcal{H}}}}$ para casi todo $x \in \mathcal{X}$.
Esto permite interpretar a $\Phi$ como una aplicación a un espacio de \textit{características},
más aún, en este espacio $k(\cdot,\cdot)$ actúa como un producto interno, esto se
concreta en el siguiente teorema.

\begin{theorem}[Aplicación kernel de Mercer]
\label{mercer_app}
Si $k$ es un kernel que cumple las condiciones del teorema (\ref{mercer}), se
puede construir una aplicación $\Phi$ a un espacio donde $k$ se comporta
como un producto interno:
\begin{equation*}
    k(x,x') = \langle \Phi(x), \Phi(x') \rangle
\end{equation*}
para casi todo $x,x' \in \mathcal{X}$. Más aún, para todo $\varepsilon >0$, existe
una aplicación $\Phi_{n}$ a un espacio $n-$dimensional con producto interno tal que
\begin{equation*}
    |k(x,x') - \langle \Phi_{n}(x), \Phi_{n}(x') \rangle| < \varepsilon
\end{equation*}
para casi todo $x,x' \in \mathcal{X}$.
\end{theorem}

La convergencia uniforme en el teorema anterior implica que para cualquier precisión
$\varepsilon >0$, debe existir un $n \in \mathbb{N}$ tal que $k$ puede ser aproximado
como un producto interno en $\mathbb{R}^{n}$. Lo anterior se observa al notar que
para casi todo $x,x' \in \mathcal{X}$, se tiene $|k(x,x') - \langle \Phi^{n}(x), \Phi^{n}(x') \rangle| < \varepsilon$,
donde $ \Phi^{n}(x): x \mapsto (\sqrt{\lambda_1}\psi_1(x), \ldots,\sqrt{\lambda_n}\psi_n(x))$.
En tal contexto, se puede interpretar al espacio de características como un espacio
finito dimensional dentro de cierta precisión $\varepsilon$. Esta característica
es una parte esencial en el aprendizaje con kernels y dará lugar a la técnica
conocida como \textit{truco del kernel}.

La condición de simetría para $k(\cdot, \cdot)$, necesaria en el teorema (\ref{mercer}),
permite dotar de esta propiedad al producto interno definido en (\ref{mercer_app}).
Por otra parte, es necesario caracterizar la positividad de $k(\cdot, \cdot)$ de
manera tal que se pueda contextualizar dentro del aprendizaje de máquinas, para ello
se utiliza la noción de matriz de \textbf{Gram}.

\begin{definition}[Matriz de Gram]
\label{eq:3}
Dado un un kernel $k(\cdot, \cdot)$ y un conjunto de puntos $\mathcal{X}_n=\left \lbrace x_i ~|~ i=1,...,n \right \rbrace$, se define la matriz de Gram como la matriz $K$ que verifica $K_{i,j}=k(x_i,x_j)$.
\end{definition}

Esta noción, permite definir la positividad de un kernel $k(\cdot,\cdot)$
en función de las propiedades que sus matrices de Gram poseen, en este sentido,
el concepto de \textbf{matriz semidefinida} juega un rol fundamental.
\begin{definition}[matriz semidefinida positiva]
\label{eq:1}
Una matriz real $K$ de $n$ x $n$ se dice  semidefinida positiva (denotado SDP) si para todo vector  v $\in \mathbb{R}^{n}$ se cumple que $Q$(v)$=  $v$^\top K $v  $\geq 0$. Si además se cumple que $Q($v$)=0 \Leftrightarrow$ v $=0$, entonces se dice que la matriz $K$ es definida positiva( denotada DP).
\end{definition}
Dentro de las propiedades de este tipo de matrices se encuentra la siguiente proposición:
\begin{proposition}
\label{eq:2}
Una matriz simétrica es semidefinida positiva si y solo si todos sus valores propios son no negativos.
\end{proposition}
Finalmente, se podrá caracterizar la positividad de un kernel $k(\cdot, \cdot)$ mediante la siguiente proposición:
\begin{proposition}
	\label{eq:3}
	Un kernel $k$ es semidefinido positivo si y sólo si toda matriz de Gram $K$ generada con $k$ es semidefinida positiva. En ese caso, se dice que $k$ es una función de covarianza.
\end{proposition}
En particular, esto significa que cualquier función simétrica y semidefinida
positiva, acepta una descomposición por medio del teorema de Mercer y por tanto
induce una transformación $\Phi$ a un espacio de características (posiblemente
de dimensión infinita, $N_{\mathcal{H}}=\infty$) donde actúa como un producto interno.
Si por otra parte, se posee una transformación $\Phi: \mathcal{X} \rightarrow \mathcal{H}$,
donde $\mathcal{H}$ es un espacio con producto interno, es posible construir un
kernel por medio de $k(x,x')=\langle \Phi(x), \Phi(x') \rangle_{\mathcal{H}}$.
Este es por definición, simétrico (lo hereda del producto interno) y además
para todo $c_i \in \mathbb{R}$, $x_i \in \mathcal{X}$, $i=1,...,m$, cumple
\begin{equation*}
\label{eq:5}
\sum_{i,j}c_i c_j k(x_i,x_j)=\left \langle  \sum_{i} c_i \Phi(x_i) , \sum_{j} c_j \Phi(x_j)    \right \rangle = \left \| \sum_{i} c_i \Phi(x_i) \right \|^2 \geq 0
\end{equation*}
Por tanto será semidefindo positivo y finalmente un kernel que cumple las condiciones
del teorema (\ref{mercer}). Esto permite modelar kernels por medio de transformaciones
conocidas, a la vez que permite asumir la existencia de una transformación dado un kernel.
\section{Espacios de Hilbert con kernel reproductor - RKHS}
Hasta este punto, se puede entender un kernel como una función semidefinida positiva
y simétrica compatible con una representación, ya sea implícita o explicita,
de un espacio inicial $\mathcal{X}$ en un espacio con producto interno (o \textit{pre-Hilbertiano}).
El propósito de esta sección corresponde a estudiar las características de este espacio
con el fin de obtener herramientas para el uso de kernels en tareas de aproximación
de funciones.

\begin{definition}[Espacio de Hilbert con Kernel Reproductor]
	\label{eq:6}
Sea $\mathcal{X}$ conjunto no vacíoy $\mathcal{H}$ un espacio de Hilbert
de funciones $f: \mathcal{X} \rightarrow \mathbb{R}$. Se denomina a
$\mathcal{H}$ espacio de Hilbert con kernel reproductor con producto punto
$\left \langle \cdot ,  \cdot \right \rangle $ (y por tanto la norma
$\left \| f \right \|:=\sqrt{\left \langle f,f \right \rangle})$ si existe
una función $k:\mathcal{X} \times \mathcal{X} 	\rightarrow \mathbb{R}$
con las siguientes propiedades.
\begin{enumerate}
	\item $k(\cdot,\cdot)$ tiene la propiedad de reproductor \\
	$\left \langle f, k(x, \cdot) \right \rangle = f(x)$ para todo $f \in \mathcal{H}$, en particular\\
  se verifica: $	\left \langle k(x, \cdot) ,k(x', \cdot) \right \rangle =k(x,x')$

  \item $k(\cdot,\cdot)$ genera $\mathcal{H}$, es decir, $\mathcal{H}=\overline{gen \left \lbrace k(x,\cdot)~ | ~ x \in \mathcal{X} \right \rbrace}$ donde  $\overline{X}$ denota la completación del conjunto $X$.
\end{enumerate}
\end{definition}

Por definición (punto 2), un RKHS es un espacio formado por combinaciones lineales
de funciones de la forma $k_{x}(\cdot)$ para todo $x \in \mathcal{X}$ y sus funciones
limite (clausura). Esto último quiere decir, que toda función $f \in \mathcal{H}$
puede ser aproximada a través de sucesiones de combinaciones lineales de elementos
de la forma $k_{x}(\cdot)$, que claramente están únicamente determinados por el
kernel $k(\cdot, \cdot)$.

Por su parte, la propiedad de reproducción (punto 1), se puede comprender al observar
que para todo $f \in \mathcal{H}$, espacio RKHS correspondiente al kernel $k(\cdot, \cdot)$,
es posible escribir $f$ como:
\begin{equation}
    f(\cdot) = \sum_{i = 1}^{\infty} \alpha_i k(x_i, \cdot)
\end{equation}

Que por el teorema de Mercer verifica:
\begin{equation}
    f(x) = \sum_{i = 1}^{\infty} \alpha_i \sum_{j = 1}^{N_{\mathcal{H}}} \lambda_j \psi_j(x_i)\psi_j(x)
\end{equation}

Como todo funcional de la forma $k_x(\cdot) = k(x,\cdot) \in \mathcal{H}$ y este espacio posee producto interno, es posible escribir $\left \langle f(\cdot), k(x, \cdot) \right \rangle$ de la forma:

\begin{align}
 \left \langle f, k_x \right \rangle_{\mathcal{H}} &=\sum_{i=1}^{\infty}\alpha_i \sum_{j=1}^{N_\mathcal{H}}\sum_{l=1}^{N_\mathcal{H}} \lambda_j \psi_j(x_i)\lambda_l \psi_{l}(x) \left \langle \psi_j, \psi_{l} \right
 \rangle_{L_2(\mathcal{X})} \\
 &=
\sum_{i=1}^{\infty} \alpha_i \sum_{j=1}^{N_\mathcal{H}} \lambda_j (x_i )  \psi_j  (x_i )  \psi_j(x) \\
  &=f(x )
\end{align}

Donde en la primera ecuación se utiliza la otrogonalidad $\psi_i \perp \psi_j$ en todo $i \neq j$. Posteriormente, se redefine el producto
interno en $L_2{(\mathcal{X})}$ de manera que $\langle \psi_j , \psi_l \rangle = \delta_{ij}/\lambda_j$, finalmente se tiene que
\footnote{$\delta_{ij} = 1$ si $i = j$, y $\delta_{ij} = 0$ en caso
contrario. Este objeto matemático se conoce como \textit{delta de
Kronecker}.} :
\begin{equation}
  	\left \langle f, k(x, \cdot) \right \rangle = f(x)
\end{equation}

para todo $f \in \mathcal{H}$. Es decir, en un espacio compuesto por
funciones generadas por un kernel de Mercer se cumple la propiedad de
reproducción, por lo tanto, si el espacio es completo (en caso contrario
se considera la adherencia del conjunto) se posee un RKHS con respecto a tal
kernel. Desde este punto vista, se puede también concluir que es posible
construir un espacio RKHS con respecto a un espacio $\mathcal{X}$ y un
kernel de Mercer, defiendo el producto interno de las funciones generadas
por el kernel de manera tal que se cumpla la propiedad de reproducción.
Finalmente, se consideran las funciones limite de tal conjunto para construir
el RKHS correspondiente.

Por definición para cada espacio RKHS existe una función kernel, por el argumento
anterior, se puede intuir la existencia de cierta relación de equivalencia entre
los kernels que describen a un espacio RKHS. La respues a tal inquietud es aún más
fuerte, pues establece que a cada RKHS corresponde una único kernel \cite{Aro1950}.

\begin{theorem}[Moore-Aronszajn]
Sea $\mathcal{X}$ un conjunto, para cada función definida positiva $k$
en $\mathcal{X} \times \mathcal{X}$ existe un único RKHS asociado a
esta. Más aún, cada RKHS posee un único kernel definido positivo
asociado.
\end{theorem}

Debido a esta relación de unicidad, tiene sentido definir al espacio
RKHS $\mathcal{H}$ por medio de la base ortogonal conformada por las
funciones propias de su kernel.
\begin{proposition}
  Sea $\mathcal{H}$ el espacio compuesto por las
combinaciones lineales de las funciones propias ortogonales
$\lbrace \psi_j  \rbrace_j^{N_{\mathcal{H}}}$ de un kernel $k(\cdot,\cdot)$ con respecto a una medida $\mu$. Toda $f \in \mathcal{H}$, se expresa por tanto como $f ( x ) = \sum _ { i = 1 } ^ { N_{\mathcal{H}} } f _ { i } \psi _ { i } ( x )$, si se impone la restricción, $ \sum _ { i = 1 } ^ { N } \frac { f _ { i } ^ { 2 } } { \lambda _ { i } } < \infty$ , es posible dotar a $\mathcal{H}$ con el producto interno $\langle f , g \rangle _ { \mathcal { H } } = \sum _ { i = 1 } ^ { N } \frac { f _ { i} q _ { i } } { \lambda _ { i } }$, donde $\lambda_j$ es el valor propio asociado a $\psi$. Si $\mathcal{H}$ es completo, entonces es el RKHS asociado a $k(\cdot,\cdot)$.
\end{proposition}

De esta forma, se puede formular el siguiente teorema:

\begin{theorem}[Teorema representador]
Dado un kernel $k$ sobre  $\mathcal{H}$, seconsidera el espacio de
funciones:
\begin{equation*}
\mathcal { H } _ { 0 } = \left\{ f ( x ) = \sum _ { i = 1 } ^ { n _ { f } } f _ { i } k \left( x , x _ { i } \right) | n _ { f } \in \mathbb { N } , x _ { i } \in \mathcal { X } , f _ { i } \in \mathbb { R } \right\}
\end{equation*}
con el producto interno $\langle f , g \rangle _ { \mathcal { H } } = \sum _ { i = 1 } ^ { n _ { f } } \sum _ { j = 1 } ^ { n _ { g } } f _ { i } g _ { j } k \left( x _ { i } , x _ { j } \right)$. Luego $\mathcal{H} = \overline{\mathcal{H}}_0$ es el RKHS asociado a $k(\cdot,\cdot)$.
\end{theorem}

Lo anterior es directo de $f ( x ) = k _ { \overline { x } } ( x ) \in \mathcal { H } _ { 0 } , \text { donde } n _ { f } = 1 , x _ { 1 } = \overline { x } \text { y } f _ { i } = 1$, de donde se obtiene:
\begin{equation*}
 \left\langle f , k _ { \overline { x } } \right\rangle _ { \mathcal { H } _ { 0 } }  = \sum _ { i = 1 } ^ { n _ { f } } f _ { i } k \left( \overline { x } , x _ { i } \right) = f ( \overline { x } )
\end{equation*}

Dado que $k(\cdot,\cdot)$ es único con respecto a su RKHS.

\section{Kernels}

En la presente sección se estudian distintos kernels. Se observan sus
características y se derivan modelos basados en estas.

\subsection{Kernel RBF}

El \textit{kernel exponencial cuadrático} (SE) o \textit{gaussiano} para $\bm{x} = (x_1, \ldots, x_D)^T \in \mathbb{R}^D$ se define por:
\begin{equation}
\label{eq:14.2.1}
k \left( \bm{x},\bm{x'} \right) = \exp \left( - \frac{1}{2} \left( \bm{x}-\bm{x' } \right) ^\top \Sigma^{-1} \left( \bm{x}-\bm{x' } \right) \right)
\end{equation}

donde si  $\Sigma$ es diagonal, se reduce a
\begin{equation}
k \left( \bm{x},\bm{x'} \right) = \exp \left( - \frac{1}{2} \sum_{j=1}^{D} \frac{1}{ \sigma_{j}^2  }    \left(    x_j - x'_j     \right) ^2  \right)
\end{equation}

En este tipo de kernels, se puede interpretar $\sigma_j$ como el
\textbf{ancho de banda característico} de la componente
(característica) $j$-ésima. En tal contexto, si $\sigma_{j}$=$\infty$,
los valores de tal dimensión pasan a ser ignorados. Por lo tanto, se
pueden identificar las características o componentes con menor
influencia examinando la magnitud de su ancho de banda correspondiente.
Este método se conoce como \textbf{Kernel ARD}.

Por otra parte, si $\Sigma$ es esférica, se obtiene el kernel isotrópico
\begin{equation}
\label{funcionradialbase}
k \left(\bm{x},\bm{x'} \right)= \exp \left( - \frac{ \| \bm{x} -\bm{x'} \| ^2  }{2 \sigma^2} \right)
\end{equation}

donde $\sigma$ Se conoce como \textbf{ancho de banda}. Las funciones
del tipo (\ref{funcionradialbase}), se conocen como \textbf{funciones de base radial}, y se caracterizan por depender unicamente de la diferencia entre los puntos que opera, es decir, $k(\bm{x},\bm{x}') = k(\bm{x} - \bm{x}')$. Por este motivo el kernel de la ecuación
(\ref{funcionradialbase}) se conoce como kernel RBF (Radial Basis
Function).

\subsection{Kernels lineales}

Como se estudió en la sección pasada, la obtención de la aplicación de
características $\Phi(\cdot)$ asociada a un kernel requería de una
descomposición espectral. Sin embargo, el proceso de obtener un kernel
desde a partir de una aplicación de características requiere únicamente de:
\begin{equation*}
k \left( \bm{x},\bm{x'} \right)= \phi (\bm{x}) ^\top	\phi (\bm{x'}) = \left \langle \phi (\bm{x}) , \phi (\bm{x'}) \right \rangle
\end{equation*}

El caso $\phi = I_d$, es decir $\phi (\bm{x})=\bm{x}$ para todo $x$, se
obtiene el \textbf{kernel lineal}, este corresponde al producto
interno del espacio donde se opera. En el caso $\mathcal{X} = \mathbb{R}^D$ este se define de manera natural por:

\begin{equation*}
k \left( \bm{x},\bm{x'}\right)= \bm{x}^\top \bm{x'}
\end{equation*}

\subsection{Matern Kernels}

Los \textbf{Matern kernel }, corresponden funciones del tipo,
\begin{equation*}
\label{eq:14.2.5}
k(r)= \frac{2^{(1-v)}}{\Gamma(v)}	\left(    \frac{\sqrt{2v}r}{l} \right) ^v K_v 	\left(    \frac{\sqrt{2v}r}{l} \right)
\end{equation*}

Donde $r=\|\bm{x}-\bm{x'}\|$ , $v >0$, $l >0 $ y $K_v$ corresponde a
una función de Bessel modificada. Cuando $v \rightarrow \infty $ este kernel se aproxima al kernel SE. Por otra parte, si $v= \frac{1}{2}$ el kernel se simplifica a
\begin{equation*}
k(r)= exp(-r/ l)
\end{equation*}

\section{Kernels derivados de modelos probabilísticos generativos}

En presencia de un modelo generativo probabilístico de vectores
representados por $p(\bm{x}|\theta)$. Es posible utilizar la estructura
de tal modelo para construir funciones kernel.

\subsection{Kernels producto de probabilidad}
En el contexto anterior, se define un kernel por medio de la siguiente
identidad
\begin{equation}
\label{eq:kernel}
k(\bm{x_i},\bm{x_j})= \int p(\bm{x} | \bm{x_i})^\rho p(\bm{x} | \bm{x_j})^\rho  dx
\end{equation}

donde $\rho>0$ y $p(\bm{x} | \bm{x_i})$ se aproxima por $p(\bm{x} |
\widehat{\theta} (\bm{x_i}))$, donde $\widehat{\theta} (\bm{x_i}))$ es
un parámetro estimado obtenido usando el dato $\bm{x_i}$. Este kernel
es llamado \textbf{kernel producto de probabilidad} \cite{Jebara2004}.

El modelo ajustado será usado para ver cuan similar son los objetos que
se operan. En particular, si se ajusta el modelo a $\bm{x_i}$ se desea
que este sea capaz de indicar cuando otro dato $\bm{x_j}$ es similar.
Si a modo de ejemplo, se supone $p(\bm{x}| \bm{\theta)}=\mathcal{N} (\mu , \sigma^2 \bm{I})$ donde $\sigma^2$ es fijo. Con $\rho=1$ y
$\widehat{\mu}(\bm{x_i})=\bm{x_i}$ y $\widehat{\mu}(\bm{x_j})=\bm{x_j}$,
el kernel

\begin{equation}
k(\bm{x_i},\bm{x_j})=\frac{1}{(4\pi \sigma^{2})^{D/2}} \exp \left( - \frac{1}{(4\pi \sigma^{2})}  \|\bm{x_i}-\bm{x_j} \|^2 \right)
\end{equation}

el cual corresponde a kernel de base radial.
Se puede calcular la ecuación \ref{eq:kernel} por una variedad de modelos generativos, incluyendo aquellos con variables latentes.

\subsection{Kernels Fisher}

Una manera más eficiente para usar modelos generativos en definición
kernels es usar \textbf{Kernel Fisher} \cite{Jaakk1998} definido por
\begin{equation*}
k(\bm{x}, \bm{x' })=\bm{g}(\bm{x})^ \top  \bm{F}^{-1} \bm{g}(\bm{x'})
\end{equation*}
donde $\bm{g}$ es el gradiente de la log verosimilitud o \textbf{vector
de puntaje}, evaluado en el estimador MLE $\widehat{\theta}$:
\begin{equation*}
\bm{g}(\bm{x}) \triangleq \nabla_\theta \log p(\bm{x} | \theta )|_{\widehat{\theta}}
\end{equation*}

$\bm{F}$ corresponde a la matriz de información de Fisher, la cual es
corresponde esencialmente a la matriz Hessiana:
\begin{equation*}
\bm{F}= \nabla \nabla \log p(\bm{x} | \theta )|_{\widehat{\theta}}
\end{equation*}

Debido a que el estimador $\widehat{\theta}$ se obitne en función de
todos los datos, se tiene que similitud de $\bm{x}$ y $\bm{x'}$ es
por tanto calculada usando todos los datos.

Si $\bm{g}(\bm{x})$ es la dirección en la cual los datos $\bm{x}$
condicionan el valor de $\widehat{\theta})$ al momento de maximizar la
verosimilitud. Se desea considerar como similares a los puntos $\bm{x}$ y $\bm{x'}$ si las direcciones para el gradiente que condicionan la verosimilitud en $\theta$ son similares entre si.

\section{Kernels en modelos lineales generalizados}

Con las herramientas estudiadas, es posible definir una clase de
modelos basados en kernel para tareas de clasificación y regresión.

\subsection{Máquinas de kernel}

Se define una \textbf{Máquina de kernel} como un modelo lineal
generalizado (GLM) donde el vector de características tiene la forma:
\begin{equation}
\label{kernelised}
\bm{\phi}(\bm{x})=[ k( \bm{x} , \mu_1),..., k (\bm{x} , \mu_K)]
\end{equation}
donde $ \mu_k \in \mathcal{X}$ son un conjunto de $K$ \textbf{centroides}. Si $k$ es un kernel RBF, este modelo se denota por una red RBF.
El vector de características presente en la ecuación \ref{kernelised}
se denota como \textbf{vector de características kernelisado}. En este
enfoque, no se requiere que el kernel sea Mercer.

Se puede usar el vector de características kernelizado entrenar un
modelo de regresión logística definiendo $p(y| \bm{x} , \theta)= Ber(\bm{w}^\top \phi(\bm{x}) )$. Esta formulación permite definir
bordes de descición no lineales de manera sencilla a traves de la elección del kernel.

De la misma forma, es posible usar el vector de características
anterior para entrenar un modelo de regresión lineal al definir
$p(y| \bm{x} , \theta)= \mathcal{N} ( \bm{w}^\top \phi(\bm{x}) , \sigma^2 )$.

\subsection{LIVMs, RVMs y Maquinas de vectores sparse}

Al formular modelos basados en Maquinas de kernel, no se especifica la
manera optima para la obtención de centroides $\mu_k$. Un enfoque
consiste en encontrar clusters en los datos y luego asignarlos como
prototipos. Sim embargo, no se puede garantizar que a través de esta
búsqueda se encuentren representaciones convenientes para la
predicción. A esto se suma la elección del número de clusters, que
tampoco esta especificada. Otro enfoque es considerar hacer cada
elemento $\bm{x_i}$ como un prototipo, de esta forma se obtiene
\begin{equation*}
\bm{\phi}(\bm{x})=[ k( \bm{x} , \bm{x}_1),..., k (\bm{x} , \bm{x}_N)]
\end{equation*}
De esta forma, se tienen tantos parámetros con datos. Si bien esto
supone un problema desde el punto de vista computacional, es posible
subsanarlo por medio del uso de distribuciones prior $\bm{w}$ que
induzcan dispersión (sparsity) en el regresor final, de esta forma, no
solo se resuelve el problema referente a la alta cantidad de
parámetros, sino que también la dispersión inducida en el vector de
características permite escoger aquellos puntos más relevantes,
permitiendo seleccionar un subconjunto de los datos para predecir. Los
modelos que hace uso de este principio se denominan \textbf{Maquinas de vectores sparse}.

En tal sentido, la elección más natural al momento de inducir
dispersión es el uso de regularización $l_1$ \cite{Kri2005}. A la maquina de vectores resultante del uso de este regularizador para inducir dispersión se le denominará \textbf{LIVM}  (Maquina de vectores regularizada $l_1$) . Análogamente, el uso de  un regularizador $l_2$  da lugar a \textbf{L2VM} (máquina de vectores $l_2$-regularizada), en este último modelo no se induce dispersión (debido al tipo de regularización).

Es también posible implementar el método ARD para inducir dispersión,
el método resultante se denomina \textbf{Máquina de vectores de relevancia} o \textbf{RVM} \cite{Tip2001}.

Otra manera maquina de vector sparse los lo modelos conocidos como
\textbf{máquinas de vectores de soporte} (SVM)

\subsection{Maquinas de vectores de soporte (SVM)}

Este m\'etodo se desarroll\'o para resolver problemas de
clasificaci\'on y aproximaci\'on est\'atica de funciones \cite{Cor1995}
\cite{Ste2008}, tanto para problemas de regresi\'on como para reconocimiento
de patrones. La formulaci\'on de esta metodolog\'ia, adaptada a la
estimaci\'on no lineal de funciones, comienza considerando una
regresi\'on de la forma:
\begin{equation}
f(x)=w^T \varphi(x) + b
\label{eq:1_svm}
\end{equation}
En este contexto, la aplicaci\'on $ \varphi: \mathbb{R}^n \rightarrow
\mathbb{R}^{N_{\mathcal{H}}} $ corresponde la transformaci\'on de caracterisitcas
sobre los datos. El parametro $ w \in \mathbb{R}^{N_{\mathcal{H}}} $ corresponde a al vector de pesos asociado a la construcci\'on del modelo y cuya dimensi\'on depende directamente de $ \varphi $, pudiendo ser infinito dimensional.

Se busca estimar minimizando el riesgo emp\'irico $ \textbf{R}_{emp}$  definido por:

\begin{equation}
\textbf{R}_{emp}={\frac{1}{N} \sum_{i=1}^{N} \left| {y_i -w^T \varphi{(x_i)}-b} \right|_{\varepsilon}}
\end{equation}

Donde se emplea la funci\'on de perdida $ \varepsilon $-insensible de Vapnik, definida por:

\begin{equation}
\left| {y - f(x)} \right|_{\varepsilon}=
\begin{cases}
0,  &\text{si } |y-f(x)|\leq \varepsilon  \\
|y-f(x)|-\varepsilon,  &\text{en caso contratrio}
\end{cases}
\end{equation}

Una primera aproximaci\'on a un modelo de entrenamiento, viene dada por el siguiente problema de optimizaci\'on:

\begin{equation}
\label{pr:problema_simple}
\left \lbrack{
\begin{aligned}
\boxed{P}:~&\underset{w,b}{\text{min}}~ J_P(w) = \frac{1}{2}w^T w \\
\text{s.a}~~~ & y_i - w^T\varphi(x_i)-b \leq \varepsilon, \; i = 1, \ldots, N~ \\
& w^T\varphi(x_i)+b - y_i \leq \varepsilon, \; i = 1, \ldots, N~
\end{aligned}
} \right \rbrack
\end{equation}

Dicha formulaci\'on corresponde al caso en que todos los datos de
entrenamiento se encuentran dentro una banda de ancho $ \varepsilon $.
Como es de esperar, elegir una precisi\'on muy alta puede dejar puntos
de entrenamiento fuera de dicha banda, lo cual hace que el problema
(\ref{pr:problema_simple}) se vuelva infactible. Para solucionar tal
inconveniente y obtener un modelo m\'as flexible, se agregan variables
de holgura $\xi_i $ , $ \xi^*_i ~$ para $ i=1, \ldots,N $ , de esta
manera se permiten ciertos puntos fuera de la precisi\'on
preestablecida, al agregar dichas variables, se obtiene el siguiente
problema:

\begin{equation}
\label{pr:problema_holgura}
\left \lbrack{
	\begin{aligned}
	\boxed{P}:~&\underset{w,b,\xi,\xi^*}{\text{min}}~ J_P(w,\xi,\xi^*) = \frac{1}{2}w^T w + c \sum_{i=1}^{N}(\xi_i + \xi^*_i)\\
	\text{s.a}~~~ & y_i - w^T\varphi(x_i)-b \leq \varepsilon +\xi_i, \; i = 1, \ldots, N~ \\
	& w^T\varphi(x_i)+b - y_i \leq \varepsilon +\xi_i^*, \; i = 1, \ldots, N~ \\
	&  \xi_i,\xi^*_i \geq 0, \; k=1, \ldots,N ~
	\end{aligned}
} \right \rbrack
\end{equation}

Donde la constante $ c>0 $ determina la proporci\'on de \textit{desviaciones} toleradas a partir de cierta precisi\'on $ \varepsilon $.
El Lagrangiano de este problema es:

\begin{equation}
\label{eq:Lagrange}
\begin{aligned}
&\mathcal{L}(w,b,\xi,\xi^*;\alpha,\alpha^*,\eta,\eta^*)=\\
&\frac{1}{2}w^T w + c \sum_{i=1}^{N}(\xi_i + \xi^*_i) - \sum_{i=1}^{N} \alpha_i (\varepsilon +\xi_i-y_i+w^T \varphi(x_i)+b) \\
&-\sum_{i=1}^{N} \alpha^*_i (\varepsilon +\xi^*_i+y_i-w^T\varphi(x_i)+b)-\sum_{i=1}^{N}(\eta_i \xi_i + \eta^*_i \xi^*_i)
\end{aligned}
\end{equation}

Donde los factores $ \alpha_i,\alpha_i^*,\eta_i,\eta_i^* \geq 0 $ son los multiplicadores de Lagrange correspondientes. El punto estacionario de
(\ref{eq:Lagrange}) se caracteriza de la siguiente manera:

\begin{equation}
\underset{\alpha,\alpha^*,\eta,\eta^*}{\text{max}} ~ \underset{w,b,\xi,\xi^*}{\text{min}~} \mathcal{L}(w,b,\xi,\xi^*;\alpha,\alpha^*,\eta,\eta^*)
\end{equation}

De las condiciones de optimalidad se observa:

\begin{equation}
\left \lbrack
\begin{aligned}
\frac{\partial\mathcal{L}}{\partial w}=0 &\rightarrow w=\sum_{i=1}^{N} (\alpha_i - \alpha^*_i)\varphi(x_i) \\
\frac{\partial\mathcal{L}}{\partial b}=0 &\rightarrow \sum_{i=1}^{N} (\alpha_i - \alpha^*_i) = 0 \\
\frac{\partial\mathcal{L}}{\partial \xi_i}=0 &\rightarrow c-\alpha_i-\eta_i=0\\
\frac{\partial\mathcal{L}}{\partial \xi^*_i}=0 &\rightarrow c-\alpha^*_i-\eta^*_i=0
\end{aligned}
\right.
\end{equation}

De lo cual se obtiene el siguiente problema dual:

\begin{equation}
\label{pr:Dual_1}
\left \lbrack{
\begin{aligned}
\boxed{D}:~ \underset{\alpha,\alpha^*}{\text{max}}~ J_D(\alpha,\alpha^*) = &-\frac{1}{2} \sum_{i,j=1}^{N}(\alpha_i-\alpha^*_i)(\alpha_j-\alpha_j^*)\varphi(x_i)^T\varphi(x_j) \\
&-\varepsilon\sum_{i=1}^{N}(\alpha_i+\alpha^*_i) + \sum_{i=1}^{N}y_i(\alpha_i - \alpha^*_i)\\
\text{s.a}~~~~~~~~~~~~~~ & \sum_{i=1}^{N}(\alpha_i - \alpha^*_i)=0 \\
&\alpha_i,\alpha^*_i \in [0,c], ~ i=1,...,N
\end{aligned}
} \right \rbrack
\end{equation}

Finalmente la representaci\'on dual del modelo de regresi\'on es:

\begin{equation}
\label{eq:rep_dual_n_iernel}
f(x)=\sum_{i=1}^{N}(\alpha_i-\alpha^*_i)\varphi(x_i)^T\varphi(x)+b
\end{equation}

Los datos de entrada $ x_i $ tales que la diferencia $ \alpha_i -\alpha^*_i$ es no nula, son los llamados \textit{vectores de soporte}.
La estructura de (\ref{eq:rep_dual_n_iernel}) permite hacer la
regresi\'on s\'olo sobre estos elementos, por lo que este método induce
o dispersi\'on en la estimación (\ref{eq:rep_dual_n_iernel}).

Esta formulación responde a un esquema de regresión no lineal donde de
induce dispersión por medio de restricciones, la formulación presentada
presenta la gran desventaja de ser incalculable cuando la dimensión del
espacio de características $N_{\mathcal{H}}$ es infinita. Usando los
resultados obtenidos anteriormente, es posible reformular por medio de
la herramienta conocida como el \textbf{truco del kernel}
\cite{Hof2008}.

El truco del kernel consiste en reemplazar el calculo de productos
internos por una función kernel haciendo uso del resultado
(\ref{eq:3}). En el caso de la formulación SVM, es posible reemplazar
$\varphi(x_i)^T\varphi(x_j)$ por una función kernel de la forma
$k(x_i,x_j) = \varphi(x_i)^T\varphi(x_j)$. La estructura de este kernel
no se conoce a piori, esto pues la definición de $\varphi(\cdot)$ fue
netamente teórica. Sin embargo, se sabe que su producto interno se
puede representar por un kernel. La idea detrás de este método
consiste en seleccionar un kernel de manera tal que las funciones base
de su RKHS subyacente permitan una aproximación lo suficientemente
flexible y regular.

 Al aplicar el truco del kernel a los modelos de entrenamiento y regresi\'on obtenidos anteriormente para SVM, se obtiene la versi\'on \textit{kernelizada}:

\begin{equation}
\label{pr:Dual_ier}
\left \lbrack{
	\begin{aligned}
	\boxed{D}:~ \underset{\alpha,\alpha^*}{\text{max}}~ J_D(\alpha,\alpha^*) = &-\frac{1}{2} \sum_{i,j=1}^{N}(\alpha_i-\alpha^*_i)(\alpha_j-\alpha_j^*)k(x_i,x_j) \\
	&-\varepsilon\sum_{i=1}^{N}(\alpha_i+\alpha^*_i) + \sum_{i=1}^{N}y_i(\alpha_i - \alpha^*_i)\\
	\text{s.a}~~~~~~~~~~~~~~ & \sum_{i=1}^{N}(\alpha_i - \alpha^*_i)=0 \\
	&\alpha_i,\alpha^*_i \in [0,c], ~ i=1,...,N
	\end{aligned}
} \right \rbrack
\end{equation}

Con su respectiva formula de regresi\'on dual:

\begin{equation}
\label{eq:rep_dual_kernel}
f(x)=\sum_{i=1}^{N}(\alpha_i-\alpha^*_i)k(x_i,x)+b
\end{equation}

\subsection{LS-SVM}
Una variante del método SVM es su versión basada en minimos cuadrados,
la cual se formula como una alternativa más eficiente\cite{Suy2002}.

Su formulaci\'on es similar a la expuesta en \ref{pr:problema_holgura}, considerando la formula habitual de regresi\'on presentada en (\ref{eq:1_svm}):
\begin{equation}
\label{pr:LSSVM_simple}
\left \lbrack{
	\begin{aligned}
	\boxed{P}:~&\underset{w,b,e}{\text{min}}~ J_P(w,e) = \frac{1}{2} w^T w + \gamma \frac{1}{2} \sum_{i=1}^{N}e_{i}^2 \\
	\text{s.a}~~~ & y_i = w^T\varphi(x_i) + b + e_i ~,~~i = p, \ldots, N~ \\
	\end{aligned}
} \right \rbrack
\end{equation}
En esta nueva formulaci\'on, los t\'erminos de error $ e_i $ juegan un papel similar al de las variables de holgura $ \xi_i,~\xi^*_i $ en (\ref{pr:problema_holgura}), con excepci\'on de que la funci\'on de perdida que se aplica a estas variables es ahora cuadr\'atica. Se comienza entonces por establecer el Lagrangiano:

\begin{equation}
\label{eq:Lagrange_LS}
\mathcal{L}(w,b,e;\alpha)= J_P(w,e) -\sum_{i=1}^{N} \alpha_i (w^T \varphi(x_i)+b - e_i - y_i)
\end{equation}

Donde los coeficientes $ \alpha_i $ son los multiplicadores de Lagrange. Las condiciones de optimalidad obtenidas son las siguientes:

\begin{equation}
\label{eq:cond_Lagrange_LSSVM}
\left \lbrack
\begin{aligned}
\frac{\partial\mathcal{L}}{\partial w}=0 &\rightarrow w=\sum_{i=1}^{N} \alpha_i \varphi(x_i) &\\
\frac{\partial\mathcal{L}}{\partial b}=0 &\rightarrow \sum_{i=1}^{N} \alpha_i = 0 &\\
\frac{\partial\mathcal{L}}{\partial e_i}=0 &\rightarrow \alpha_i=\gamma e_i & i=1,...,N\\
\frac{\partial\mathcal{L}}{\partial \alpha_i}=0 &\rightarrow w^T\varphi(x_i)+b+e_i-y_i=0 & i=1,...,N
\end{aligned}
\right.
\end{equation}

Al despejar $ w $, $ e_i $ y reemplazar en la ultima igualdad se obtiene:

\begin{equation}
\label{eq:pre_matriz}
 \sum_{i=1}^{N} \alpha_i \varphi(x_i)^T \varphi(x_j)+b+ \frac{\alpha_j}{\gamma} = y_j ~~~ j=1,...,N
\end{equation}

Usando el truco del kernel en la forma:

\begin{equation}
\label{eq:piv1}
\mathbf{K}_{ij} = k(x_i,x_j)= \varphi(x_i)^T \varphi(x_j) ~~~ j=1,...,N
\end{equation}

Reescribiendo (\ref{eq:pre_matriz}) y haciendo uso de la notaci\'on $\mathbf{y}=[y_1 , ..., y_{N}]^T,~\mathbf{1}_{N} = [1,...,1]^T$, $\boldsymbol{\alpha}=[\alpha_1,...,\alpha_{N}]^T$, la condici\'on $ \sum_{i=1}^{N}{\alpha_i}=0$ obtenida en (\ref{eq:cond_Lagrange_LSSVM}) y el resultado (\ref{eq:piv1}), se elabora el siguiente problema:

\begin{equation}
\label{pr:Dual_LSSVM}
\left \lbrack{
	\begin{aligned}
\boxed{D}:~&\text{Resolver para} ~ \boldsymbol{\alpha},~b \\
&\left[
\begin{array}{c|c}
0 & \mathbf{1}_{N}^T \\
\hline
\mathbf{1}_N & \mathbf{K}+{\gamma}^{-1}{\mathbf{I}}
\end{array}
\right]
\left[{
\begin{array}{c}
b \\
\hline
\boldsymbol{\alpha}
\end{array}
}\right] = \left[{
\begin{array}{c}
0 \\
\hline
\mathbf{y}
\end{array}
}\right]
\end {aligned}
}\right \rbrack
\end{equation}

Haciendo nuevamente uso de las condiciones obtenidas en (\ref{eq:cond_Lagrange_LSSVM}), se puede verificar que el estimador obtenido es bastante similar en estructura al expuesto en (\ref{eq:rep_dual_kernel}), siendo este:
\begin{equation}
\label{eq:rep_dual_LSSVM}
f(x)=\sum_{i=1}^{N}\alpha_i k(x,x_i) + b
\end{equation}

El proceso de entrenamiento a trav\'es de (\ref{pr:Dual_LSSVM}), se ve facilitado num\'ericamente, ya que se basa en la soluci\'on de un sistema de ecuaciones lineales, el cual presenta una soluci\'on \'unica cuando es de rango completo y que adem\'as, en su forma dual, no depende de la dimensi\'on del espacio de caracter\'isticas. Estas ventajas, sin embargo, no son gratuitas, pues como es posible observar en la obtenci\'on de las condiciones de primer orden del Lagrangiano (\ref{eq:cond_Lagrange_LSSVM}), la ecuaci\'on $ \alpha_i = \gamma e_i $ anula la caracter\'istica de \textit{dispersi\'on} que posee la formulaci\'on tradicional, esto se debe a que en este nuevo escenario, se dificulta la existencia de $ \alpha_i $ exactamente iguales a $ 0 $, dando el car\'acter de \textit{vector de soporte} a todos los elementos presentes en el conjunto de entrenamiento, cada uno con cierto factor $ \alpha_i $ de influencia en el modelo.

\subsection{Clasificación kernel knn}

En un clasificador 1-NN se necesita calcular la distancia Euclidiana
de un vector test a todos los puntos en entrenamiento, para luego  encontrar al mas cercano, y asignar su etiqueta. Esto puede ser kernelizado al observar:

\begin{equation}
\label{Eq:vecinomascercano}
\| x_i x_{i'} \|_2 ^2= \left \langle x_i,x_i \right \rangle  + \left \langle x_{i'},x_{i'} \right \rangle - 2\left \langle x_{i},x_{i'} \right \rangle
\end{equation}

Esta reformulación permite la aplicación del clasificador knn una clase
más amplia de problemas.

\subsection{Clustering de K-medoides Kernelizado}

El clustering K-medias al, igual que knn, usa la distancia euclidiana
para medir similitud.

Para kernelizar este algoritmo, se reemplaza el algoritmo K-medias con el algoritmo \textbf{K-medoides}. El cual es similar a K-medias, pero en vez de representar cada centroide del cluster por la media de todos los vectores de datos asignados a este cluster, se hace que cada centroide sea uno de los vectores de datos por si mismo. En este caso se busca resolver:
\begin{equation*}
m_k= \argmin_{i: z_i = k} \sum_{i': z_{i'} = k}d(i,i')
\end{equation*}
donde
\begin{equation*}
d(i,i') \triangleq  \| \bm{x_i} - \bm{x_i'} \|_{2}^{2}
\end{equation*}

Esto toma $ O(n_k ^2)$ trabajo por cluster, mientras K-medias toma $O(n_k D)$ para actualizar cada cluster. En el algoritmo (\ref{alg:5}) se aprecia la estrucutra del método. Al calular el medoide más cercano a la clase, el problema se transforma en \textbf{clasificación medoide mas cercano}\cite{Has2009}
Este algoritmo puede ser kernelizado usando Ecuación (\ref{Eq:vecinomascercano}) para reemplazar la distancia $ d(i,i')$ .

\begin{algorithm}
\textbf{Inicializar}: $m_1,\ldots,m_K$ subconjunto aleatorio de tamaño $K$ en $\lbrace 1,\ldots, N \rbrace$\;
\DontPrintSemicolon
\Repeat{
$z_i = \argmin_{k} d(i,m_k)$ para $i = 0, \ldots,N$\;
$m_k \leftarrow \argmin_{i: z_i = k} \sum_{i': z_{i'} = k}d(i,i')$\;
}
\textbf{until} convergencia
\caption{Algorimo kernel K-medoids}\label{alg:5}
\end{algorithm}


\subsection{Kernel PCA}

PCA requiere encontrar vectores propios de la matriz de covarianza muestral $\bm{S}= \frac{1}{N} \sum_{i=1}^{N} \bm{x_i} \bm{x_i}^\top = (1/N)  \bm{X}^\top \bm{X} $. Sin embargo, se puede también calcular PCA, encontrando los vectores propios  de la matriz de producto interno $\bm{X}\bm{X}^\top$, esto permitirá derivar el método \textbf{kernel PCA}\cite{Sch1998}.

Primero, sea $\bm{U}$ la matriz ortogonal conteniendo los vectores propios de $\bm{X} \bm{X}^\top$ con los correspondientes valores propios en $\Lambda$. Por definición, se tiene $(\bm{X} \bm{X}^\top) \bm{U}=  \bm{U} \Lambda$. Premultiplicando por $\bm{X}^\top$:
\begin{equation*}
(\bm{X}^\top \bm{X})   (\bm{X}^\top \bm{U})=(\bm{X}^\top \bm{U})\Lambda
\end{equation*}
de donde se ve que los vectores propios de $(\bm{X}^\top \bm{X})$
son $ \bm{V}= \bm{X}^\top \bm{U}$ con los valores propios dados por  $\Lambda$. Sin embargo, tales vectores propios no se encuentran normalizados. Dado $ \|v_j\|^2= \bm{u_j}^\top \bm{X} \bm{X}^\top \bm{u_j}= \lambda_j  \bm{u_j}^\top  \bm{u_j}= \lambda_j$.
por lo que es posible obtener un vector propio normalizado por medio de $\bm{V_{pca}}=\bm{X}^\top \bm{U} \Lambda^{-1/2} $.

Sea ahora $ \bm{K}= \bm{X} \bm{X}^\top $  la matriz de Gram del kernel $k(\cdot,\cdot)$. Sea $\Phi$ la matriz diseño correspondiente a la transformación de características inducida por $k(\cdot,\cdot)$. $\bm{S_{\phi}}= \frac{1}{N} \sum_{i} \bm{\phi_i} \bm{\phi_i}^{\top}$ la correspondiente matriz
de covarianza en el espacio de características. Los vectores propios
son dados por $\bm{V_{kpca}}=\bm{\Phi}^\top \bm{U} \Lambda^{-1/2} $ donde $\bm{U}$ y $\Lambda$ contienen los vectores
propios y valores propios de $\bm{K}$. Claramente no es posible calcular
$\bm{V_{kpca}}$ ,dado que $\bm{\Phi_i}$ es potencialmente de
dimensión infinita. Sin embargo, se puede calcular la proyección de un vector $\bm{x}$	sobre el espacio característica como sigue.
\begin{equation*}
\phi _ { * } ^ { T } \mathbf { V } _ { k p c a } = \phi _ { * } ^ { T } \mathbf { \Phi } \mathbf { U } \mathbf { \Lambda } ^ { - \frac { 1 } { 2 } } = \mathbf { k } _ { * } ^ { T } \mathbf { U } \mathbf { \Lambda } ^ { - \frac { 1 } { 2 } }
\end{equation*}
donde $\mathbf { k } _ { * } = \left[ k \left( \mathbf { x } _ { * } , \mathbf { x } _ { 1 } \right) , \ldots , k \left( \mathbf { x } _ { * } , \mathbf { x } _ { N } \right) \right]$.

Hasta este punto, se asume que los datos proyectados tienen media cero,
lo cual en general no es el caso. Para sortear tal detalle, se define el vector de características centrado como $ \tilde { \boldsymbol { \phi } } _ { i } = \phi \left( \mathbf { x } _ { i } \right) - \frac { 1 } { N } \sum _ { j = 1 } ^ { N } \boldsymbol { \phi } \left( \mathbf { x } _ { j } \right)$. La matriz Gram de vectores característicos centrados esta dada por:

\begin{align*}
\tilde { K } _ { i j } & = \tilde { \phi } _ { i } ^ { T } \tilde { \phi } _ { j } \\
& = \phi _ { i } ^ { T } \phi _ { j } - \frac { 1 } { N } \sum _ { k = 1 } ^ { N } \phi _ { i } ^ { T } \phi _ { k } - \frac { 1 } { N } \sum _ { k = 1 } ^ { N } \phi _ { j } ^ { T } \phi _ { k } + \frac { 1 } { N ^ { 2 } } \sum _ { k = 1 } ^ { N } \sum _ { l = 1 } ^ { M } \phi _ { k } ^ { T } \phi _ { l } \\
\end{align*}
De donde

\begin{equation*}
\tilde { K } _ { i j } = k \left( \bm { x } _ { i } , \bm { x } _ { j } \right) - \frac { 1 } { N } \sum _ { k = 1 } ^ { N } k \left( \bm { x } _ { i } , \bm { x } _ { k } \right) - \frac { 1 } { N } \sum _ { k = 1 } ^ { N } k \left( \bm { x } _ { j } , \bm { x } _ { k } \right) + \frac { 1 } { N ^ { 2 } } \sum _ { k = 1 } ^ { N } \sum _ { l = 1 } ^ { M } k \left( \bm { x } _ { k } , \bm { x } _ { l } \right)
 \
\end{equation*}
Lo cual puede ser expresado en notación matricial como:
\begin{equation*}
  \tilde { \mathbf { K } } = \mathbf { H K H }
\end{equation*}
donde $\mathbf { H } \triangleq \mathbf { I } - \frac { 1 } { N } \mathbf { 1 } _ { N } \mathbf { 1 } _ { N } ^ { T }$ es la \textbf{matriz centrada}. El proceso seguido anteriormente, se
resume en el algoritmo \ref{alg:9}.

\begin{algorithm}
\DontPrintSemicolon
\KwIn{$\bm{K}$ de tamaño $N \times N$, $\bm{K}_{*}$ de tamaño $N_{*}\times N$, numero de dimenciones latentes L}
$\bm{O} = \mathbf { 1 } _{N} \mathbf { 1 }_{N}^T /N$ \;

$\tilde{ \mathbf { K } } = \bm{K} - \bm{O}\bm{K} - \bm{K}\bm{O} + \bm{O}\bm{K}\bm{O}$\;

$[\bm{U}, \bm{\Lambda}] = \text{eig}\tilde{ \mathbf { K } }$\;
\For{$i=1,\ldots,N$}{

$\bm{v}_i = \bm{u}_i / \sqrt{\lambda_i}$

}
$\bm{O}_{*} = \mathbf { 1 } _{N_{*}} \mathbf { 1 }_{N}^T /N$ \;

$\tilde{ \mathbf { K } } = \bm{K}_{*} - \bm{O}_{*}\bm{K}_{*} - \bm{K}_{*}\bm{O}_{*} + \bm{O}_{*}\bm{K}_{*}\bm{O}_{*}$\;

$Z = \tilde{ \mathbf { K } } \bm{V}(:,1:L)$;
\caption{kernel PCA}\label{alg:9}
\end{algorithm}

\section{Kernels para construir modelos generativos}

Otra case kernels, conocidos como kernels de suavizado, permiten
caracterizar estimaciones no paramétricas de desidades. Lo cual puede
ser utilizado para generar estimaciones de densidades de manera no
supervisada, así como también para crear modelos generativos tanto para
regresión como para clasificación.

\subsection{Kernel de suavizado}

Un \textbf{kernel suavizado} es una función de un argumento que satisface las siguientes propiedades:
\begin{equation*}
\int k ( x ) d x = 1 , \int x k ( x ) d x = 0 , \int x ^ { 2 } k ( x ) d x > 0
\end{equation*}
Un ejemplo sencillo es el \textbf{kernel gaussiano}:
\begin{equation*}
k ( x ) \triangleq \frac { 1 } { ( 2 \pi ) ^ { \frac { 1 } { 2 } } } e ^ { - x ^ { 2 } / 2 }
\end{equation*}
En este caso, se puede controlar el ancho de banda del kernel
introduciendo un parámetro $h$:
\begin{equation*}
k _ { h } ( x ) \triangleq \frac { 1 } { h } k \left( \frac { x } { h } \right)
\end{equation*}
Esto puede ser generalizado a un kernel  RBF:
\begin{equation*}
k _ { h } ( \mathbf { x } ) = k _ { h } ( \| \mathbf { x } \| )
\end{equation*}
No obstante, en el caso del kernel gaussiano, el kernel de suavizado pasa a ser:
\begin{equation*}
k _ { h } ( \mathbf { x } ) = \frac { 1 } { h ^ { D } ( 2 \pi ) ^ { D / 2 } } \prod _ { j = 1 } ^ { D } \exp \left( - \frac { 1 } { 2 h ^ { 2 } } x _ { j } ^ { 2 } \right)
\end{equation*}
Si bien los kernels de la familia gausiana poseen propiedades que permiten simplicar el gasto computacional, su soporte no es acotado, por lo que en modelos reales carece de expresividad. Una alternatica a esto es el \textbf{kernel Epanechnikov}, el cual posee soporte compacto y esta definido por:
\begin{equation*}
k ( x ) \triangleq \frac { 3 } { 4 } \left( 1 - x ^ { 2 } \right) \mathbb { I } ( | x | \leq 1 )
\end{equation*}

Desafortunadamente, el kernel Epanechnikov no es diferenciable en los bordes del soporte. Una alternativa es el \textbf{kernel tri-cubo}, defino como sigue:
\begin{equation*}
k ( x ) \triangleq \frac { 70 } { 81 } \left( 1 - | x | ^ { 3 } \right) ^ { 3 } \mathbb { I } ( | x | \leq 1 )
\end{equation*}
Este tiene soporte compacto y tiene dos derivadas continuas en los bordes del soporte.
Por otra parte, se puede obtener un kernel a partir de la distribución uniforme:
\begin{equation*}
k ( x ) \triangleq \mathbb { I } ( | x | \leq 1 )
\end{equation*}

Denominado \textbf{kernel boxcar}.

\subsection{kernel density estimation (KDE)}

El modelo GMM estudiado anteriormente corresponde a un estimador de densidad paramétrico para datos en $\mathbb { R } ^ { D }$. Este requiere especificar el número de componentes $K$ y ubicaciones $\bm{\mu_k}$ de los cluster. Una alternativa para la estimación de $\bm{\mu_k}$ es ubicar un cluster central por punto de datos, tal que $\boldsymbol { \mu } _ { i } = \mathbf { x } _ { i }$. En este caso el modelo se convierte:
\begin{equation*}
p ( \mathbf { x } | \mathcal { D } ) = \frac { 1 } { N } \sum _ { i = 1 } ^ { N } \mathcal { N } ( \mathbf { x } | \mathbf { x } _ { i } , \sigma ^ { 2 } \mathbf { I } )
\end{equation*}
Lo cual se puede generalizar escribiendo:
\begin{equation*}
\hat { p } ( \mathbf { x } ) = \frac { 1 } { N } \sum _ { i = 1 } ^ { N } k _ { h } \left( \mathbf { x } - \mathbf { x } _ { i } \right)
\end{equation*}
Modelo conocido como estimador \textbf{Parzen window density} , o \textbf{estimador de densidad de kernel (KDE)} , y corresponde a una formulación no paramétrica.  La ventaja sobre modelos paramétricos es que no se requiere el modelo ajustado y no existe la necesidad de elegir $K$. La desventaja es que el modelo toma mucho almacenamiento de memoria y gran tiempo para evaluar.

\subsection{De KDE a KNN}

Se puede usar KDE para definir las clases de densidades condicionales en un clasificador generativo.  Esto proporciona una derivación  alternativa  del clasificador KNN. En KDE con un kernel  boxcar se fijara el ancho de banda y se contará cuantos puntos caen dentro del hyper cubo centrado sobre cada dato. Si se modela el ancho de banda $h$ como un parámetro de modelo, se permite que el ancho de banda o volumen sea distinto para cada dato. Específicamente, se aumentará el volumen alrededor de $\bm{x}$ hasta encontrar $K$ datos, sin importar su clase. Sea el volumen resultante $V$ con $V(x)$ y $N_c(x)$ ejemplos de clase $c$ en tal volumen. Luego es posible estimar la densidad condicional la de clase de como sigue:
\begin{equation*}
p ( \mathbf { x } | y = c , \mathcal { D } ) = \frac { N _ { c } ( \mathbf { x } ) } { N _ { c } V ( \mathbf { x } ) }
\end{equation*}
Donde $N_c$ es el número total de ejemplos en la clase $c$ en aquel conjunto de datos. La clase a priori puede ser modelada como
\begin{equation*}
p ( y = c | \mathcal { D } ) = \frac { N _ { c } } { N }
\end{equation*}
Por lo tanto, la posterior de clase esta dada por:
\begin{equation*}
p ( y = c | \mathbf { x } , \mathcal { D } ) = \frac { \frac { N _ { c } ( \mathbf { x } ) } { N _ { c }V ( \mathbf { x } ) } \frac { N _ { c } } { N } } { \sum _ { c ^ { \prime } } \frac { N _ { c '}  ( \mathbf { x } ) } { { N _ { c ^ { \prime } }  V( \mathbf { x } ) }} \frac { N _ { c ^ { \prime } } } { N }  } = \frac { N _ { c } ( \mathbf { x } ) } { \sum _ { c ^ { \prime } } N _ { c ^{ \prime }  } ( \mathbf { x } ) } = \frac { N _ { c } ( \mathbf { x } ) } { K }
\end{equation*}
Donde se usa el hecho que $\sum _ { c } N _ { c } ( \mathbf { x } ) = K$, dado que se elige un total de $K$ puntos alrededor de cada punto.

\subsection{Regresión por kernel}

Es posible usar KDE para regresión. El objetivo es calcular la esperanza condicionada
\begin{equation*}
f ( \mathbf { x } ) = \mathbb { E } [ y | \mathbf { x } ] = \int y p ( y | \mathbf { x } ) d y = \frac { \int y p ( \mathbf { x } , y ) d y } { \int p ( \mathbf { x } , y ) d y }
\end{equation*}
Se puede usar KDE para aproximar la densidad conjunta 	$	p ( \mathbf { x } , y )$ como sigue
\begin{equation*}
p ( \mathbf { x } , y ) \approx \frac { 1 } { N } \sum _ { i = 1 } ^ { N } k _ { h } \left( \mathbf { x } - \mathbf { x } _ { i } \right) k _ { h } \left( y - y _ { i } \right)
\end{equation*}
Por lo tanto
\begin{equation*}
\left.\begin{aligned} f ( \mathbf { x } ) & = \frac { \frac { 1 } { N } \sum _ { i = 1 } ^ { N } k _ { h } \left( \mathbf { x } - \mathbf { x } _ { i } \right) \int y k _ { h } \left( y - y _ { i } \right) d y } { \frac { 1 } { N } \sum _ { i = 1 } ^ { N } k _ { h } \left( \mathbf { x } - \mathbf { x } _ { i } \right) \int k _ { h } \left( y - y _ { i } \right) d y } \\ & = \frac { \sum _ { i = 1 } ^ { N } k _ { h } \left( \mathbf { x } - \mathbf { x } _ { i } \right) y _ { i } } { \sum _ { i = 1 } ^ { N } k _ { h } \left( \mathbf { x } - \mathbf { x } _ { i } \right) } \end{aligned} \right.
\end{equation*}

 Lo cual se obtiene al observar que la integral  $\int k _ { h } \left( y - y _ { i } \right) d y = 1$. Ademàs del hecho que $\int y k _ { h } \left( y - y _ { i } \right) d y = y _ { i }$. Esto sigue definiendo $ x = y - y _ { i }$  y usando la propiedad de media igual a cero de kernel suavizado:
\begin{equation*}
\int \left( x + y _ { i } \right) k _ { h } ( x ) d x = \int x k _ { h } ( x ) d x + y _ { i } \int k _ { h } ( x ) d x = 0 + y _ { i } = y _ { i }
\end{equation*}
Se puede reescribir el resultado anterior de la siguiente manera:
\begin{equation*}
\left.\begin{aligned} f ( \mathbf { x } ) & = \sum _ { i = 1 } ^ { N } w _ { i } ( \mathbf { x } ) y _ { i } \\ w _ { i } ( \mathbf { x } ) & \triangleq \frac { k _ { h } \left( \mathbf { x } - \mathbf { x } _ { i } \right) } { \sum _ { i ^ { \prime } = 1 } ^ { N } k _ { h } \left( \mathbf { x } - \mathbf { x } _ { i ^ { \prime } } \right) } \end{aligned} \right.
\end{equation*}

La prediccón es por tanto una suma ponderada de los resultados de los datos de entrenamiento, donde los pesos dependen de que tan similar es $\mathbf{ x }$ de los puntos de entrenamientos almacenados. Este método es llamado \textbf{regresión kernel}, \textbf{kernel suavizado} o el modelo \textbf{Nadaraya-Watson}.

\subsection{Regresión localmente ponderada}
Si se define $k _ { h } \left( \mathbf { x } - \mathbf { x } _ { i } \right) = k \left( \mathbf { x } , \mathbf { x } _ { i } \right)$ se puede reescribir la predicción hecha mediante regresiòn kernel como sigue
\begin{equation*}
\hat { f } \left( \mathbf { x } _ { * } \right) = \sum _ { i = 1 } ^ { N } y _ { i } \frac { k \left( \mathbf { x } _ { * } , \mathbf { x } _ { i } \right) } { \sum _ { i ^ { \prime } = 1 } ^ { N } k \left( \mathbf { x } _ { * } , \mathbf { x } _ { i ^ { \prime } } \right) }
\end{equation*}
Si en este caso $k \left( \mathbf { x } , \mathbf { x } _ { i } \right)$ fuera un kernel suavizado, se podrìa ignorar el termino normalización, en tal caso:
\begin{equation*}
\hat { f } \left( \mathbf { x } _ { * } \right) = \sum _ { i = 1 } ^ { N } y _ { i } k \left( \mathbf { x } _ { * } , \mathbf { x } _ { i } \right)
\end{equation*}
Este modelo consiste esencialmente en una función fija localmente constante. Este modelo se puede mejorar fijando un modelo de regresión lineal para cada punto $\mathbf { X } _ { * }$ resolviendo:
\begin{equation*}
\min _ { \beta \left( \mathbf { x } _ { * } \right) } \sum _ { i = 1 } ^ { N } k \left( \mathbf { x } _ { * } , \mathbf { x } _ { i } \right) \left[ y _ { i } - \boldsymbol { \beta } \left( \mathbf { x } _ { * } \right) ^ { T } \boldsymbol { \phi } \left( \mathbf { x } _ { i } \right) \right] ^ { 2 }
\end{equation*}
donde $\phi ( \mathbf { x } ) = [ 1 , \mathbf { x } ]$. Esto es llamado \textbf{regresión ponderada localmente}. Un ejemplo de tal metodo es \textbf{LOESS} o \textbf{LOWES} que significa para  “locally-weighted scatterplot smoothing”.
Se puede calcular el parámetro $\boldsymbol { \beta } \left( \mathbf { x } _ { * } \right)$ para cada caso de test resolviendo el siguiente problema de mínimos cuadrados ponderados:
\begin{equation*}
\boldsymbol { \beta } \left( \mathbf { x } _ { * } \right) = \left( \mathbf { \Phi } ^ { T } \mathbf { D } \left( \mathbf { x } _ { * } \right) \mathbf { \Phi } \right) ^ { - 1 } \mathbf { \Phi } ^ { T } \mathbf { D } \left( \mathbf { x } _ { * } \right) \mathbf { y }
\end{equation*}
donde $\Phi$ es una  matriz de diseño de $N \times ( D + 1 )$ y $\mathbf { D } = \operatorname { diag } \left( k \left( \mathbf { x } _ { * } , \mathbf { x } _ { i } \right) \right)$.La predicción correspondiente tiene la forma
\begin{equation*}
\hat { f } \left( \mathbf { x } _ { * } \right) = \phi \left( x _ { * } \right) ^ { T } \beta \left( \mathbf { x } _ { * } \right) = \left( \mathbf { \Phi } ^ { T } \mathbf { D } \left( \mathbf { x } _ { * } \right) \Phi \right) ^ { - 1 } \mathbf { \Phi } ^ { T } \mathbf { D } \left( \mathbf { x } _ { * } \right) \mathbf { y } = \sum _ { i = 1 } ^ { N } w _ { i } \left( \mathbf { x } _ { * } \right) y _ { i }
\end{equation*}

El termino $w _ { i } \left( \mathbf { x } _ { * } \right)$  combina el kernel local suavizado con el efecto de la regresión lineal, es llamado el \textbf{kernel equivalente} .
